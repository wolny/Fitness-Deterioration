
\chapter{Introduction}
\label{Introduction} 

\begin{quotation}
 It is impossible for any optimization algorithm to outperform random walks on all possible problems.
\end{quotation}
\begin{flushright}
 ... a conclusion from No Free Lunch Theorem
\end{flushright}

\section{A statement of a problem}

TODO: motivation of heuristics methods

A Global Optimization Algorithm is defined as optimization algorithm that
employs measures that prevent convergence to local
optima and increase the probability of finding a global optimum. However,
there are classes of problems in which
instead of finding the global optimum we are interested in finding many
local optimums whose basins of attraction are properly wide
and deep.
One way to achieve this goal is to perform several runs of a evolutionary
algorithm and alter the fitness function in every subsequent runs of the
algorithm
in a way that prevents exploration of basins which were found in
previous runs of the algorithm.
This work tries to find an effective fitness deterioration technique in
high-dimensional domain spaces by interpolating fitness landscape in the area of
basin of attraction for further fitness deterioration. Very often fitness function is computationally intensive and in such case it
is unacceptable to perform classical interpolation of the fitness function.
Making the assumption that clusters of population obtained after the single
run of the algorithm are good estimators of basins of attraction, it would
be better to exploit spatial characteristics of clusters and approximate
basins of attraction by multidimensional Gaussain functions performing
only a few fitness evaluation. Our goal is to create efficient method for
fitness deterioration using the above schema and to analyze the relation
between the deterioration accuracy and reproduction operators (mutation,
crossover, etc.).

\section{Definition of terms}

\section{Review of literature}
TODO: sequential niching, prof. Obuchowicz works

